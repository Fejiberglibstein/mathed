\documentclass[10pt, letterpaper]{article}
\usepackage{graphicx} % Required for inserting images
\usepackage{amsmath}

\usepackage{mathrsfs}  % For cursive L

\title{Iteration of Functions}
\author{emeryllium, LooserRIP, Fejiberglibstein}
\date{February 2024}

\newcommand \iter[2] {_{#1} {#2} }
\newcommand \iterator[1] {\overset{#1}{\mathscr{L}} }
 % import macros to use

\begin{document}

\maketitle

\begin{abstract}
    Defining distinct notation for and finding generalizations of the iterations of certain functions.
\end{abstract}

\tableofcontents

\newpage

\section{Introduction}
Functions can be "iterated" by plugging them into themselves. For example, $f(x)$ iterated is $f(f(x))$. This definition can be extended to find the Iterator of a function, which denotes how many times that function is iterated.

\section{Notation}
An Iterator is a generalized equation for iterating a function $n$ times. An Iteration is a solution to an iterator given specific $x$ and $n$ values.

\subsection{$\iterator{}$ notation}
In L notation, Iterators are notated as
    \footnote{Iterators can be notated with or without brackets around the function. Standard procedure is to use brackets if the expression contains multiple terms. If no $n$ value is written, assume $n=2$.}

$$ \iterator{n} f(x) lx$$
where $n$ is the iteration variable, $f(x)$ is the function to iterate, and $rx$ is the variable with respect to which you iterate. For example, $\mathscr{L}^n [2x+b] rx$
iterates with respect to $x$, not $b$, so $\mathscr{L}^3 [2x+b] rx$ would evaluate to $8x+7b$.

\subsubsection{Iterations}
Iterations are notated as
$$\mathscr{L}^n_a f(x) rx$$
where $a$ is the value at which to evaluate the Iterator. For example, $\mathscr{L}^3_0 [2x+b] rx$ evaluates to $7b$.

\subsubsection{At infinity}
Some Iterators or Iterations can be evaluated with an $n$ value of infinity. Ex:
$$\mathscr{L}^\infty [\frac{1}{2}x - 1] rx = -2$$

\subsection{Shorthand}
For the sake of brevity, Iterators of the function $f(x)$ may be notated as
$$f(x)[n]$$ 
where $n$ is the number of iterations. For example, $f(x)[3]$ evaluates to $f(f(f(x)))$. $f(x)$ can be replaced with an expression if the expression contains only one variable. Ex:
$$(2x-1)[3] = 8x-7$$
This notation does not extend to Iterations of functions.

\section{Rules of iteration}
ignore this for now until we find some actually useful rules
$$\mathscr{L}^{n_1}[\mathscr{L}^{n_2} f(x) rx] rx = \mathscr{L}^{n_1n_2} f(x) rx$$
$$f(f(x)[n_1])[n_2] = f(x)[n_1+n_2]$$

\section{Trivial generalizations}
Any function composed of one or two adjacent hyperoperations\footnote{See https://en.wikipedia.org/wiki/Hyperoperation.} (Ex. Addition and Multiplication) can be solved trivially.

\subsection{$x$}
In this function, every value {$x$} we plug in, we'll get {$x$} back. so we see that no matter how many iterations we do, we'll keep the same variable.
\\This draws us to the conclusion that 
$$\mathscr{L}^n [x] rx = x$$

\subsection{$x+k$}
Perhaps the simplest actual function, x+k is pretty simple to work out, let's imagine k = 2.
\\
n=1 (the original equation), would be {$(x+2)$}
\\
n=2 (the first iteration) would be {$(x+2)+2$}, which is {$x+4$}
\\
n=3 (the second iteration) would be {$((x+2)+2)+2$}, which is {$x+6$}
\\
\\
and quickly we already got the pattern, $\mathscr{L}^n [x+2] rx = x+2n$.
\\
substituting 2 for k, we'll get
$$\mathscr{L}^n [x+k] rx = x+kn$$

\subsection{$bx+k$}
Let's go one step further and generalize {$bx+k$}.
\\This time we'll do this without substitution, to see the variables more clearly.
\\n=1 would give us the original equation, {$bx+k$}
\\n=2, we plug {$bx+k$} inside itself and get {$b(bx+k)+k$} which simplifies to {$xb^{2}+kb+k$}
\\n=3, we repeat this and get {$b(xb^{2}+kb+k)+k$} which simplifies to {$xb^{3} + kb^{2} + kb + k$}
\\We can now notice the pattern. for any value {$n$}, we add {$xb^{n}$} and a descending sum from n-1 to 0 of {$kb^{r}$}
\\All together this comes to $$xb^{n}+\sum_{r=0}^{n-1} kb^{r}$$ % try not to use r as an iteration variable bc of the rx thing ok ty bbg
\\Which simplifies to
$$\mathscr{L}^n [bx+k] rx = \frac{k(b^{n}-1)}{b-1}+b^{n}x$$

\subsection{$bx^k$}
This time we'll generalize {$bx^k$}.
\\n=1 would give us the original equation, {$bx^{k}$}
\\n=2, we get {$b(bx^{k})^{k}$} which simplifies to {$b^{k+1}x^{k^{2}}$}
\\n=3, we get {$b(b^{k+1}x^{k^{2}})+k$} which simplifies to {$b^{k^{2} + k + 1} x^{k^{3}}$}
\\We can now notice the pattern. for any value {$n$}, we multiply {$x^{k^{n}}$} together with {$b$} to the power of a sum from {$n-1$} to {$0$} of {$k^{r}$}
\\All together this comes to $$b^{\sum_{r=0}^{n-1} k^{r}} x^{k^{n}}$$
\\Which simplifies to
$$\mathscr{L}^n [bx^k] rx = b^{\frac{k^{n}-1}{k-1}}x^{k^{n}}$$

\subsection{$(bx)^k$}

\newpage
\section{Non-trivial generalizations}

\subsection{$\frac{1}{x+b}$}
\begin{equation}
    \begin{split}
        \mathscr{L}^1 \frac{1}{x+b} rx & = \frac{1}{x+b}\\
        \mathscr{L}^2 \frac{1}{x+b} rx & = \frac{x+b}{1+bx+b^2}\\
        \mathscr{L}^3 \frac{1}{x+b} rx & = \frac{1+bx+b^2}{x+2b+b^2x+b^3}\\
        \mathscr{L}^4 \frac{1}{x+b} rx & = \frac{x+2b+b^2x+b^3}{1+2bx+3b^2+b^3x+b^4}\\
    \end{split}
\end{equation}
By observation, the numerator of $\mathscr{L}^n \frac{1}{x+b} rx$ is the same as the denominator of $\mathscr{L}^{n-1} \frac{1}{x+b} rx$. Given this,
$$\mathscr{L}^{n} \frac{1}{x+b} rx = \frac{G_{n-1}(x)}{G_n(x)}$$
Where $G_0(x) = 1$, $G_1(x) = \frac{1}{x+b}$, $G_2(x) = \frac{x+b}{1+bx+b^2}$, etc.

\subsection{$x + \frac{1}{x}$}
\begin{equation}
    \begin{split}
        f(x)[1] = \  \frac{x^2+1}{x}\\
        f(x)[2] = \  \frac{x^3+2x}{x^2+1}\\
        f(x)[3] = \  \frac{x^4+3x^2+1}{x^3+2x}\\
        f(x)[4] = \ \frac{x^5+4x^3+3x}{x^4+3x^2+1}\\
    \end{split}
\end{equation}

This is currently *not* generalized, though we do have some information on it.

\end{document}
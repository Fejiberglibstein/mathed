\documentclass[12pt, letterpaper]{article}

\usepackage{graphicx} % Required for inserting images
\usepackage{amsmath} % Math package

\usepackage{geometry} % For changing document size

\usepackage{mathrsfs}  % For cursive L

\title{Iteration of Functions}
\author{Emeryllium, LooserRIP, Fejiberglibstein}
\date{February 2024}

\newcommand \iter[2] {_{#1} {#2} }
\newcommand \iterator[1] {\overset{#1}{\mathscr{L}} }
 % import macros to use

\begin{document}

\maketitle

\begin{abstract}
    Defining distinct notation for and finding generalizations of the iterators of certain functions.
\end{abstract}

\tableofcontents
\newpage

\section{Introduction}
Functions can be \textit{Iterated} by plugging them into themselves. For example, $f(x)$ iterated once is $f\big(f(x)\big)$. 
\\
The \textit{Iterator} of a function, $f(x)$, is a function $l(n)$ such that plugging in any value of $n$ is equal to the $n$th iteration of $f(x)$


\section{Notation}

An \textit{Iterator} is a generalized equation for iterating a function $n$ times.
An \textit{Iteration} is a solution to an iterator at a specific $x$ and $n$ value.

\subsection{$\iterator{}$ notation}
In L notation, Iterators are notated as
    \footnote{Iterators can be notated with or without brackets around the function. Standard procedure is to use brackets if the expression contains multiple terms. If no $n$ value is written, assume $n=2$.}

$$ \iterator{n} \, f(x) \, lx$$
where $n$ is the iteration variable, $f(x)$ is the function to iterate, and $lx$ is the variable with respect to which you iterate. For example, $\iterator{n} [2x+b] lx$
iterates with respect to $x$, not $b$, so $\iterator{3} [2x+b] lx$ would evaluate to $8x+7b$.

\subsubsection{At infinity}
An $n$ value of $\pm\infty$ is allowed and may even result in a finite solution. Ex:
$$\iterator{\infty} [\frac{1}{2}x - 1] lx = -2$$

\subsubsection{Iterations}
Iterations (solutions of Iterators) are written using an evaluation bar. Ex:
$$\iterator{n}f(x)lx\bigg|_{x=0}$$

\subsection{Shorthand}
For the sake of brevity, Iterators of the function $f(x)$ may be notated as
$$f(x)[n]$$ 
where $n$ is the number of iterations. For example, $f(x)[3]$ evaluates to $f(f(f(x)))$. $f(x)$ can be replaced with an expression if the expression contains only one variable. Ex:
$$(2x-1)[3] = 8x-7$$
This notation does not extend to Iterations of functions.

\section{Rules of iteration}

\subsection{Identity Rule}
For the identity function $f(x) = x$, 
\[
\iterator{n} \, x \, lx = x
\]
This is known as the identity 

ignore this for now until we find some actually useful rules
$$\iterator{n_1}[\iterator{n_2} f(x) lx] lx = \iterator{n_1\cdot n_2} f(x) lx$$
$$f(f(x)[n_1])[n_2] = f(x)[n_1+n_2]$$

\section{Trivial generalizations}
Any function composed of one or two adjacent hyperoperations\footnote{See https://en.wikipedia.org/wiki/Hyperoperation.} (Ex. Addition and Multiplication) can be solved trivially.

\subsection{$x$}
In the identity function \(f(x) = x\), every iteration of \(x\) is \(x\). 
\\
Therefore
$$\iterator{n} \, x \, lx = x$$
This is known as the \textit{Identity rule}.

\subsection{$x+k$}
The generalization of the function $f(x) = x+k$ is also trivial.
\begin{align*}
    f(x)[1] &= x + k             &= x + 1k \\
    f(x)[2] &= (x + k) + k       &= x + 2k \\
    f(x)[3] &= ((x + k) + k) + k &= x + 3k \\
       \dots \\
    f(x)[n] &= x + \sum_{i=0}^n k  &= x + nk \\
\end{align*}
The pattern $f(x)[n] = x + nk$ can easily be observed. Therefore,
$$\iterator{n} \, (x+k) \, lx = x+nk$$.
\\

\subsection{$bx+k$}
Generalizing {$f(x) = bx+k$} is as follows.
\begin{align*}
    f(x)[1] &= bx + k             &= xb^{1}+kb^{0} \\
    f(x)[2] &= b(bx + k) + k       &= xb^{2}+kb^{1}+kb^{0} \\
    f(x)[3] &= b(b(bx + k) + k) + k &= xb^{3} + kb^{2} + kb^{1} + kb^{0} \\
       \dots \\
    f(x)[n] &= xb^{n}+\sum_{r=0}^{n-1} kb^{r}  &= \frac{k(b^{n}-1)}{b-1}+b^{n}x
\end{align*}
\\
The pattern $f(x)[n]$ can be observed and generalized, therefore
$$\iterator{n} [bx+k] rx = \frac{k(b^{n}-1)}{b-1}+b^{n}x$$

\subsection{$bx^k$}
Generalizing {$f(x) = bx^{k}$} is as follows.
\begin{align*}
    f(x)[1] &= bx^{k}             &= b^{1}x^{k} \\
    f(x)[2] &= b(bx^{k}) ^{k}       &= b^{k+1}x^{k^{2}} \\
    f(x)[3] &= b(b(bx^{k})^{k})^{k} &= b^{k^{2} + k + 1} x^{k^{3}} \\
       \dots \\
    f(x)[n] &= b^{\sum_{r=0}^{n-1} k^{r}} x^{k^{n}}  &= b^{\frac{k^{n}-1}{k-1}}x^{k^{n}}
\end{align*}
\\The pattern can be observed and generalized to
$$\iterator{n} [bx^k] rx = b^{\frac{k^{n}-1}{k-1}}x^{k^{n}}$$

\subsection{$(bx)^k$}
Generalizing {$f(x) = (bx)^{k}$} is quite similar to {$f(x) = bx^k$}
\begin{align*}
    f(x)[1] &= (bx)^{k}             &= b^{k}x^{k} \\
    f(x)[2] &= (b(bx)^{k})^{k}       &= b^{k^{2}+k}x^{k^{2}} \\
    f(x)[3] &= (b(b(bx)^{k})^{k})^{k} &= b^{k^{3} + k^{2} + k} x^{k^{3}} \\
       \dots \\
    f(x)[n] &= b^{\sum_{r=1}^{n} k^{r}} x^{k^{n}}  &= b^{\frac{k(k^{n}-1)}{k-1}}x^{k^{n}}
\end{align*}
\\The pattern can be observed and generalized to
$$\iterator{n} [(bx)^k] rx = b^{\frac{k(k^{n}-1)}{k-1}}x^{k^{n}}$$

\newpage
\section{Non-trivial generalizations}

\subsection{$\frac{1}{x+b}$}
\begin{equation}
    \begin{split}
        \iterator{1} \frac{1}{x+b} lx & = \frac{1}{x+b}\\
        \iterator{2} \frac{1}{x+b} lx & = \frac{x+b}{1+bx+b^2}\\
        \iterator{3} \frac{1}{x+b} lx & = \frac{1+bx+b^2}{x+2b+b^2x+b^3}\\
        \iterator{4} \frac{1}{x+b} lx & = \frac{x+2b+b^2x+b^3}{1+2bx+3b^2+b^3x+b^4}\\
        \dots
    \end{split}
\end{equation}
By observation, the numerator of $\iterator{n} \frac{1}{x+b} lx$ is the same as the denominator of $\iterator{n-1} \frac{1}{x+b} lx$. Given this,
$$\iterator{n} \frac{1}{x+b} lx = \frac{G_{n-1}(x)}{G_n(x)}$$
Where
\begin{equation}
    \begin{split}
        G_0(x) & = 1\\
        G_1(x) & = x+b\\
        G_2(x) & = 1+bx+b^2\\
        G_3(x) & = x+2b+b^2x+b^3\\
        G_4(x) & = 1+2bx+3b^2+b^3x+b^4\\
        G_5(x) & = x+3b+3b^2x+4b^3+b^4x+b^5\\
        \dots 
    \end{split}
\end{equation}
In the function $G_n(x)$, the amount of terms is equal to $n+1$. In each subsequent term, the power of which $b$ is raised to increases by one, starting at $0$. The $b$ component of a term $t$ of $G_n(x)$ can be written as
$$b^t$$
$x$ is present on terms with odd powers of $b$ when $n$ is even, and even powers of $b$ when $n$ is odd. The $x$ component of a term $t$ of $G_n(x)$ can be written as
$$x^{\frac{1}{2}-\frac{1}{2}(-1)^{n+t}}$$
for integer values of $n$, or more generally,
$$x^{\frac{1}{2}-\frac{1}{2}\cos(\pi n+\pi t)}$$
Finally, the coefficients of the terms of $G_n(x)$ follow a discernible pattern.
\begin{equation}
    \begin{split}
        G_0 & : 1\\
        G_1 & : 1, 1\\
        G_2 & : 1, 1, 1\\
        G_3 & : 1, 2, 1, 1\\
        G_4 & : 1, 2, 3, 1, 1\\
        G_5 & : 1, 3, 3, 4, 1, 1\\
        G_6 & : 1, 3, 6, 4, 5, 1, 1\\
        \dots
    \end{split}
\end{equation}
The coefficient of term $t$ of $G_n(x)$ will follow the formula
$$\frac{\Gamma \left \lfloor \frac{n+t}{2} +1\right \rfloor}{\Gamma [t+1] \cdot \Gamma \left \lfloor \frac{n-t}{2} +1\right \rfloor}$$
where $\Gamma (z)$ is the Gamma function\footnote{See https://en.wikipedia.org/wiki/Gamma\_function}, or $(z-1)!$. The floor function can also be generalized, resulting in this expression for the coefficient of term $t$
$$\frac{\Gamma[\frac{1}{2}n+\frac{1}{2}t+\frac{1}{4}\cos[\pi n+\pi t]+\frac{3}{4}]}{\Gamma[t+1]\cdot \Gamma[\frac{1}{2}n-\frac{1}{2}t+\frac{1}{4}\cos[\pi n-\pi t]+\frac{3}{4}]}$$
Taking this as well as the other patterns of $G_n(x)$ into account, $G_n(x)$ is defined as
$$G_n(x) = \sum_{t=0}^{n}\frac{\Gamma[\frac{1}{2}n+\frac{1}{2}t+\frac{1}{4}\cos[\pi n+\pi t]+\frac{3}{4}]}{\Gamma[t+1]\cdot \Gamma[\frac{1}{2}n-\frac{1}{2}t+\frac{1}{4}\cos[\pi n-\pi t]+\frac{3}{4}]}b^{t}x^{\frac{1}{2}-\frac{1}{2}\cos(\pi n+\pi t)}$$

\subsection{$x + \frac{1}{x}$}
\begin{equation}
    \begin{split}
        f(x)[1] & = \  \frac{x^2+x^{0}}{x^{1}}\\
        f(x)[2] & = \  \frac{x^{4} + 3 x^{2} + x^{0}}{x^{3} + 1x^{1}}\\
        f(x)[3] & = \  \frac{1x^{8}+7x^{6}+13x^{4}+7x^{2}+1x^{0}}{x^{7}+4x^{5}+4x^{3}+ 1x^{1}}\\
    \end{split}
\end{equation}

This is currently *not* generalized, though we do have some information on it.

\section{Other interesting generalizations}

\subsection{Oscillators}
\subsubsection{$\sqrt{k^2-x^2}$}
The iterator for $\iterator{n} \, \sqrt{k^2-x^2} \, lx$ will oscillate between $\sqrt{k^2-x^2}$ on odd values of n and $\sqrt{x^2}$ on even values of n. The proof for this is as follows:
\paragraph{}
Assume that the domain of $f(x)$ is limited to $(-k, k)$
\begin{align*}
        f(x) &= \sqrt{k^2 - x^2} \\
        f(f(x)) &= \sqrt{k^2 - \Big( \sqrt{k^2-x^2} \Big)^2} \\
                &= \sqrt{k^2 - (k^2-x^2) } \quad | -k < x < k \\
                &= \sqrt{x^2} \quad | -k < x < k \\
        f(f(f(x))) &= \sqrt{k^2 - (\sqrt{x^2})^2} \quad | -k < x < k \\
                   &= \sqrt{k^2 - x^2} \quad | -k < x < k \\ 
\end{align*}
Since the domain of $f(x)$ was initially limited to $(-k, k)$, then the domains match. Therefore, $f(x) = f(f(f(x)))$.


\end{document}